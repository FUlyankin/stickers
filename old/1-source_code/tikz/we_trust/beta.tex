\documentclass{standalone}

\usepackage{amsmath,amsfonts,amssymb,amsthm,mathtools}  % пакеты для математики


%%%%%%%%%%%%%%%%%%%%%%%% Шрифты %%%%%%%%%%%%%%%%%%%%%%%%%%%%%%%%%

\usepackage[british,russian]{babel} % выбор языка для документа
\usepackage[utf8]{inputenc} % задание utf8 кодировки исходного tex файла

\usepackage{fontspec}         % пакет для подгрузки шрифтов
\setmainfont{Arial}   % задаёт основной шрифт документа

\usepackage{unicode-math}     % пакет для установки математического шрифта
%\setmathfont{Asana Math}      % шрифт для математики

%%%%%%%%%% Работа с картинками %%%%%%%%%
\usepackage{graphicx}                  % Для вставки рисунков
\usepackage{graphics} 
\graphicspath{{images/}{pictures/}}    % можно указать папки с картинками
\usepackage{wrapfig}                   % Обтекание рисунков и таблиц текстом


%%%%%%%%%% Работа с таблицами %%%%%%%%%%
\usepackage{tabularx}            % новые типы колонок
\usepackage{tabulary}            % и ещё новые типы колонок
\usepackage{array}               % Дополнительная работа с таблицами
\usepackage{longtable}           % Длинные таблицы
\usepackage{multirow}            % Слияние строк в таблице
\usepackage{float}               % возможность позиционировать объекты в нужном месте 
\usepackage{booktabs}            % таблицы как в книгах!  
\renewcommand{\arraystretch}{1.3} % больше расстояние между строками

% Заповеди из документации к booktabs:
% 1. Будь проще! Глазам должно быть комфортно
% 2. Не используйте вертикальные линни
% 3. Не используйте двойные линии. Как правило, достаточно трёх горизонтальных линий
% 4. Единицы измерения - в шапку таблицы
% 5. Не сокращайте .1 вместо 0.1
% 6. Повторяющееся значение повторяйте, а не говорите "то же"
% 7. Есть сомнения? Выравнивай по левому краю!

%%%%%%%%%%%%%%%%%%%%%%%% Оформление %%%%%%%%%%%%%%%%%%%%%%%%%%%%%%%%%




%%%%%%%%%%%%%%%%%%%%%%%% Графики и рисование %%%%%%%%%%%%%%%%%%%%%%%%%%%%%%%%%
\usepackage{tikz, pgfplots}  % язык для рисования графики из latex'a

\pagecolor{black}

\definecolor{gitt}{HTML}{713015}

\begin{document}
	

\begin{tikzpicture}[
        scale=2,
        dot/.style={circle,fill=black,minimum size=4pt,inner sep=0pt,
            outer sep=-1pt},
    ]
    
\begin{scope}[rotate=30]       
% Radius of regular polygons
  \newdimen\R
  \R=2cm
  \coordinate (center) at (0,0);
 \draw (0:\R)
     \foreach \x in {60,120,...,360} {  -- (\x:\R) }
              -- cycle (300:\R)
              -- cycle (240:\R)
              -- cycle (180:\R)
              -- cycle (120:\R)
              -- cycle (60:\R)
              -- cycle (0:\R)  [line width=1.9mm,color=white,fill=white,fill opacity=0];
\end{scope}
           
              
\begin{scope}[yshift=0cm]              

\draw[color=white,align=left,scale=1] (-0.2,0.65) node[right] { \fontspec{Game of Thrones KG}{\fontsize{15}{60}\textbf{IN}}};
\draw[color=white,align=left,scale=0.6] (-2.2,0.5) node[right] { \fontspec{Game of Thrones KG}{\fontsize{15}{60}\textbf{ECONOMETRICS}}};
\draw[color=white,align=left,scale=0.6] (-0.5,-0.1) node[right] { \fontspec{Game of Thrones KG}{\fontsize{15}{60}\textbf{WE}}};
\draw[color=white,align=left,scale=0.6] (-0.9,-0.7) node[right] { \fontspec{Game of Thrones KG}{\fontsize{15}{60}\textbf{TRUST}}};

\end{scope}
                  
\end{tikzpicture}

\end{document}